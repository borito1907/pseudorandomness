\documentclass{article}
\usepackage{geometry}
\usepackage{graphicx} % Required for inserting images
\usepackage{amsmath, amsthm, amssymb}
\usepackage{mathtools}
\DeclarePairedDelimiter\ceil{\lceil}{\rceil}
\DeclarePairedDelimiter\floor{\lfloor}{\rfloor}
\usepackage{parskip}
\newgeometry{vmargin={15mm}, hmargin={24mm,34mm}}
\theoremstyle{definition} 

\newtheorem{definition}{Definition}
\newtheorem{theorem}{Theorem}[section]
\newtheorem{lemma}[theorem]{Lemma}
\newtheorem{corollary}{Corollary}[theorem]
\newtheorem{proposition}[theorem]{Proposition}
\newcommand{\N}{\mathbb{N}}
\newcommand{\Z}{\mathbb{Z}}
\newcommand{\R}{\mathbb{R}}
\newcommand{\Q}{\mathbb{Q}}
\newcommand{\xtoc}{\lvert x \rvert ^{c}}
\newcommand{\bitstring}[1]{\{0,1\}^{#1}}
\newcommand{\Gen}{\textbf{Gen}}
\newcommand{\Sample}{\textbf{Sample}}
\newcommand{\Eval}{\textbf{Eval}}
\newcommand{\Invert}{\textbf{Invert}}

\DeclareMathOperator*{\E}{\mathbb{E}}

\title{Pseudorandomness}
\date{March 2024}
\author{Boran Erol}

\begin{document}

\maketitle

\section{Introduction and Motivation}

The primer to the theory of pseudorandomness is based on a fresh view 
at the \textit{question of randomness}, which has been taken by complexity
theory. The idea is to call everything random if we can't efficiently
distinguish it from the uniform random distribution. Therefore, randomness 
is not an inherent property of the object, it's dependent on the observer.

At the extreme, this approach says that the question of whether the world
is determinstic or allows for some free choice (randomness) is irrelevant.
All it matters is that how the world looks to us and computationally bounded
devices.

Three fundamental aspects of PRGs:

\begin{enumerate}
    \item The stretch
    \item The algorithms it fools (computational indistinguishability)
    \item The computational complexity of the PRG algorithm
\end{enumerate}

Three theories of randomness:

\begin{enumerate}
    \item Shannon's Information Theory: Randomness is lack of information.
    \item Kolmogorov Complexity: Randomness is lack of structure.
    \item Complexity Theoretic View
\end{enumerate}

The generation process is fixed before the distinguisher. Therefore, even though
the PRG runs in fixed polynomial time, it's indistinguishable to any
PPT distinguisher. In particular, the distinguisher is allocated more
resources than the generator. Moreover, the distinguisher is probabilistic
but the generator is deterministic.

However, in the case of derandomization, it's more natural to do the opposite.
Instead, we fix the observer first and base the generator on it.

\begin{definition}
    A deterministic polynomial time algorithm $G$ is a PRG if there's 
    a stretch function $l: \N \xrightarrow{} \N$ (satisfying $\forall k: l(k) > k$)
    such that for all PPT's D, for all positive polynomials $p$, and for all
    sufficiently large $k$ we have that 

    \[ \lvert \Pr[D(G(U_{k})) = 1] - \Pr[D(U_{l(k)}) = 1] \rvert \leq \frac{1}{p(k)}\]

    where the probability is taken over $U_{k}$ and the coin tosses of $D$.
\end{definition}

PRGs can be used to reduce the randomness complexity of randomized algorithms
without making a noticeable difference in their output.

Statistically close distributions are trivially computationally indistinguishable.

You can upgrade single-sample indistinguishability to multi(polynomial)-sample indistinguishability
by using a hybrid argument where the neighboring distributions differ by a single sample.

\newpage

\section{Pseudorandomness Workshop at Simon's Institute}

\begin{itemize}
    \item Make the probabilistic method constructive. Construct \textit{efficiently} and
    deterministically objects whose existence is guaranteed by proofs based on the probabilistic
    method. Efficiency is crucial since otherwise we get constructive proofs by enumeration.
    \item Convert a randomized algorithm for a problem of interest into a deterministic algorithm
    of \textit{comparable complexity}.
    \item \textit{Efficiently} construct deterministically (or with little true randomness)
    objects having many of the useful properties of random objects.
\end{itemize}

\begin{theorem}
    Every $n$-vertex graph has either a clique or an independent set of size
    at least $\frac{1}{2} \log n$.
\end{theorem}

\begin{theorem}
    There's an $n$-vertex graph in which the max clique
    and the maximum independent set has size at most
    $2 \log n$.
\end{theorem}

80-year-old open problem: Match Erdos's existence proof with an explicit construction.

Shannon's second theorem: If any encoding scheme works, a random encoding scheme works with
high probability.

The issue is that encoding schemes have doubly exponential sizes and it takes doubly
exponential time to decode. Lots of work in the last 70 years to make these constructions
explicit.

Derandomization of Algorithms

\begin{enumerate}
    \item Primality Testing
    \item Finding Large Primes
    \item Polynomial Identity Testing
    \item Approximating the permanent
    \item Circuit acceptance
\end{enumerate}

If the ideal PRG exists, $BPP = P$, all proofs using the probabilistic
method can be efficiently constructed and all randomized algorithms
can be randomized.

Notice that we can fool one-bit tests by just repeating a single bit.
This has a cool story behind it that's told in Minute 30 of the Fundamental
Techniques of Pseudorandomness II talk by Trevisan.

\newpage

\section{k-wise Independence}

Let $z \in F_{2}^{n}$. Let $\phi_{z}: F_{2}^{n} \xrightarrow{} F_{2}$ be the map
defined by $\phi_{a} = \langle z, a \rangle$. Then, the nullity of this map is
$n - 1$ and therefore the probability that a randomly sampled $a$ is in the kernel
of this map is 1/2.

Even simpler, this is a linear equation in $F_{2}$ so the space of solutions
is $n-1$ dimensional.

Now notice that if you have $a \neq a^{\prime}$ and consider 
$\langle z,a \rangle = 0, \langle z, a^{\prime} \rangle = 0$,
you get two linearly independent equations in $F_{2}$, 
so the space of solutions has dimension $n - 2$ no matter
whether the right-hand side is $(0,0),(0,1),(1,0),(1,1)$.

In fact, this proves that linear independence translates into stochastic
independence in this setting!

Pairwise independence can also be seen as a PRG that fools tests that only
look at two bits.

Let's now generalize this idea. Our aim is to do the following: given
$t$ random bits, we want to construct a function $h:$

For every $x \neq y$, over the randomness of $a,b$ the two values
$ax + b$ and $ay + b$ are uniform and independent.

This is a little confusing: we'll think of $x,y$ as constants and $a,b$
as variables.

Notice that this means there are $2^{n} \times 2^{n} = 2^{n+1}$ possible
values for this tuple and every value has the same probability, namely,

\[ \frac{1}{2^{n} + 1}\]

This follows from the fact that the linear system of equations given
by 

$ax + b = v$ and $ay + b = w$

has a unique solution since the two left hand-sides are linearly independent.

Thus, we can generate a function $h: \bitstring{n} \xrightarrow{} \bitstring{n}$
using $2n$ bits. This is not the most optimal, we can get it down to $n$ bits.

Notice that if $m < n$, we can use this construction and just restrict the function
to satisfy the condition.

If $m > n$, we can construct a function from $m$ bitstrings to other $m$ bitstrings
and pad all strings with 0s. Therefore, we only use $2 \times \max{m,n}$ bits.

Notice that setting $m = 1$ gives a worse pairwise generator than the Hadamard code.
In fact, $\max(m,n)$ is an optimal lower bound for $t$.

\newpage

\section{Pseudorandomness and Regularity in Graphs}

Three different regimes:

\begin{itemize}
    \item The sparse case - d = $O(1)$.
    \item The dense case - d = $\Omega(n)$.
    \item The intermediate case.
\end{itemize}



\newpage

Random graphs and functions are expanders and extractors with high probability. These are also useful
sources for constructing PRGs.

\section{Expanders}

An expander can informally be defined as a graph where every set of nodes has many neighbors.

For example, we want networks to be expander graphs since non-expanding subsets correspond to bottlenecks.

The definition of expanders is combinatorial. There are two different perspectives that are useful:
probabilistic and algebraic.

\begin{itemize}
    \item Combinatorial: Definition
    \item Probabilistic: Random walks on rapidly mixing graphs
    \item Algebraic: Second largest eigenvalue $\lambda_{2}$
\end{itemize}

Let $G = (V,E)$ be a $d$-regular graph with $n$ vertices.

We can talk about vertex expansions and edge expansions.

Notice that $h_{v} \leq h_{e} \leq d \times h_{v}$. Since we usually consider
constant degree graphs, we don't care about the distinction.

Balancing two contradicting properties: the graph should be sparse and connected
at the same time.

Another way to intuit this is as a graph that is hard to cut.

This reminded of the minimum cut and maximum flow.

\newpage

\section{Extractors}

Our goal is to get some weak source of randomness (biased randomness) and try to bits that look randomness.

Creating pure randomness from a single weak source of randomness is impossible. Thus, we explore
3 different relaxations:

\begin{itemize}
    \item Seeded Extractors
    \item Multi-source extractors
    \item Seedless extractors for structured sources
\end{itemize}



\end{document}


